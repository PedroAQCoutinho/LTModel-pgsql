\chapter{INTRODUÇÃO}
\begin{multicols}{2}

\section{Organização Geral}
O algoritmo da malha fundiária foi dividido em diferentes partes por meio de scripts no formato sql que estão ordenados segundo a ordem proc\{N1\}\_\{N2\}\_\{nome\}\_\{N3\}. Sendo: 
\begin{itemize}
    \item N1: número de 1 algarismo determinando número do grupo de processamento
    \item N2: número de 2 algarismos, número da ordem de processamento
    \item N3: número de 1 algarismo, subdividindo uma tarefa em uma ou 3 partes: \begin{itemize}
        \item Uma parte: quando o algoritmo não necessita de paralelização e o comando pode ser executado por um único processo.
        \item 3 partes: \begin{enumerate*}[label={Parte \arabic* - }]
            \item Pré configuração, cria as tabelas e índices necessários.
            \item Execução paralela, criando diversos processos que inserem paralelamente na mesma tabela.
            \item Log e consolidação dos resultados.
        \end{enumerate*}
    \end{itemize}
\end{itemize}
    
Para facilitar a execução dos scripts sql foram desenvolvidos 3 bash files, sendo os dois primeiros auxiliares e o terceiro o executável funcional. Eles podem ser executados a partir do Bash instalado pelo Git (Git Bash), ou diretamente de um bash do linux. O script bash está organizado de tal forma que é possível solicitar para rodar todo o processamento de uma vez só.

\section{Ordem de Processamento}
A ordem de processamento segue o seguinte padrão:
\begin{enumerate}
    \item Consolidação das bases do INCRA
    \item Consolidação das bases do CAR
    \item Junção da base do INCRA sobrepondo a base CAR
    \item Processa a regra de sobreposição geral
\end{enumerate}
\end{multicols}