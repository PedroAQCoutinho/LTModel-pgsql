\documentclass{book}
\usepackage[brazilian]{babel}
\usepackage[inline]{enumitem}   

\title{\bfseries {
    Documentação do executável da malha fundiária
}}
\author{Caio Hamamura}


\begin{document}
\maketitle
\thispagestyle{empty}
\clearpage

\pagenumbering{arabic}
\chapter{INTRODUÇÃO}

\section{Organização}
O algoritmo da malha fundiária foi dividido em diferentes partes por meio de scripts no formato sql que estão ordenados segundo a ordem proc\{N1\}\_\{N2\}\_\{nome\}\_\{N3\}. Sendo: 
\begin{itemize}
    \item N1: número de 1 algarismo determinando número do grupo de processamento
    \item N2: número de 2 algarismos, número da ordem de processamento
    \item N3: número de 1 algarismo, subdividindo uma tarefa em uma ou 3 partes: \begin{itemize}
        \item Uma parte: quando o algoritmo não necessita de paralelização e o comando pode ser executado por um único processo.
        \item 3 partes: \begin{enumerate*}[label={Parte \arabic* - }]
            \item Pré configuração, cria as tabelas e índices necessários.
            \item Execução paralela, criando diversos processos que inserem paralelamente na mesma tabela.
            \item Log e consolidação dos resultados.
        \end{enumerate*}
        
    \end{itemize}
\end{itemize}

\end{document}